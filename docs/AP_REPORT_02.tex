
\documentclass[a4paper, oneside, 11pt]{report}
\usepackage{epsfig,pifont,float,multirow,amsmath,amssymb}
\newcommand{\mc}{\multicolumn{1}{c|}}
\newcommand{\mb}{\mathbf}
\newcommand{\mi}{\mathit}
\newcommand{\oa}{\overrightarrow}
\newcommand{\bs}{\boldsymbol}
\newcommand{\ra}{\rightarrow}
\newcommand{\la}{\leftarrow}
\usepackage{algorithm}
\usepackage{algorithmic}
\usepackage{hyperref}
\topmargin = 0pt
\voffset = -80pt
\oddsidemargin = 15pt
\textwidth = 425pt
\textheight = 750pt
\hypersetup{
    colorlinks=true,
    linkcolor=blue,
    filecolor=magenta,      
    urlcolor=cyan,
    pdftitle={Overleaf Example},
    pdfpagemode=FullScreen,
    }
\begin{document}

\begin{titlepage}
\begin{center}
\rule{12cm}{1mm} \\
\vspace{1cm}
{\large  CMP-6048A/7009A Advanced Programming} %Delete as appropriate
\vspace{7.5cm}
\\{\Large Project Report - 10 January 2024}
\vspace{1.5cm}
\\{\LARGE MATBAP: Maths Interpreter software} % You can add to this title of modify it if you wish
\vspace{1.0cm}
\\{\Large Group members: \\ Igor Stepanenko, Lyra Dalton, Liam Farese\ }
\vspace{10.0cm}
\\{\large School of Computing Sciences, University of East Anglia}
\\ \rule{12cm}{0.5mm}
\\ \hspace{8.5cm} {\large Version 2.0}
\end{center}
\end{titlepage}


\setcounter{page}{1}
%\pagenumbering{roman}
%\newpage


\begin{abstract}
Please replace this section with your own abstract. An abstract is a brief summary (maximum 250 words) of your entire project. It should cover your objectives, your methodologies used, a brief developmental history, your final results, in particular covering the optional tasks, and a discussion and conclusion. You do not cover the literature or background in an abstract nor should you use abbreviations or acronyms. The remainder of this report template has clear chapter titles and we suggest to stick to these although you can organise your material inside each chapter to your own preferences. A guideline in size is approximately 3,500 words (not including abstract, captions and references) but no real limit on figures, tables, diagrams, pseudo-code etc.
\end{abstract}

\chapter{Introduction}
\label{chap:intro}

\section{Project statement}
This project will be a Maths Interpreter software with extensions aiming
towards a Turing complete language. It will be used to evaluate expressions, define
variables and functions, and plot functions of two variables. The software
will allow the user to enter expressions and equations using a custom language which is processed by a lexer, parser and evaluator. Plots of mathematical equations will be interactive. Additional functionality might include root finding, differentiation and integration. The software will be developed keeping in mind software engineering principles such as low coupling and high cohesion achieved through Model-View-ViewModel and Inversion of Control pattern. 

\section{Aims and objectives}
The main aim is to develop Matbap, a maths interpreter software for solving and plotting mathematical expressions. Main objectives include:
\begin{itemize}
    \item Create Lexer, Parser and Evaluator in F\# to process mathematical expressions and equations to plot.
    \item Develop a Graphical User Interface (GUI) in C\# for user interaction.
    \item Integrate F\# and C\# components keeping in mind low coupling.
\end{itemize}

\noindent % Start a new paragraph without indentation
They are broken further into subtasks in this MoSCoW table \ref{appendix:moscow} 






\chapter{Background}

Give a brief background on similar software, e.g.\ \cite{Desmos:2023}, \cite{Matlab:2023}, etc.
Also cite the books \cite{Nystrom:2021} or documentation \cite{WPF:2023} that you consulted.
You should add additional references to the corresponding bib file (References.bib) referred to in the bottom of this document.


\chapter{Development History}\label{Chap:DevHist}

linksDescribe the history of your development in terms of the iterations or sprints in your project (your Github repository or other version control should help you to retrospectively identify these). Use different sections for different sprints and subsections for specific details on the same sprint. Feel free to use subsubsections or paragraphs (which are not numbered) if needed. 

\section{Sprint 1: \href{https://liamfarese.atlassian.net/browse/AP-8}{As a user I want perform arithmetic calculations}}
\subsection{Grammar in BNF}
\begin{verbatim}
<E>    ::= <T> <Eopt>
<Eopt> ::= "+" <T> <Eopt> | "-" <T> <Eopt> | <empty>
<T>    ::= <NR> <Topt>
<Topt> ::= "*" <NR> <Topt> | "/" <NR> <Topt> | <empty>
<NR>   ::= "Num" <value> | "(" <E> ")"
\end{verbatim}

\subsection{Basic GUI}
We used WPF with C\# to develop a basic GUI - see Figure \ref{gui01}.

\begin{figure}[htb]
\begin{center}
\includegraphics[width=0.9 \columnwidth]{GUI_01.png}
\caption{A very basic GUI!}
\label{gui01}
\end{center}
\end{figure}

\subsection{Testing}
A subset of Table \ref{Table2} in Appendix \ref{app:test} could be referred to from here.

\section{Sprint 2: \href{https://liamfarese.atlassian.net/browse/AP-30}{As a user I want to have a basic GUI}}

\subsection{Updated GUI}

\subsection{Testing}

\section{Sprint 3: \href{https://liamfarese.atlassian.net/browse/AP-42}{As a user I want to be able to assign variables}}
\section{Sprint 4: \href{https://liamfarese.atlassian.net/browse/AP-50}{As a user I want to be able to plot lines and polynomials}}
\section{Sprint 5: As a developer I would like to have an AST Parser}
\section{Sprint 6: As a user I would like to run for-loops}
\section{Sprint 7: As a user I want to visualise an AST}
\section{Sprint n: And whatever we complete before we submit this}



\chapter{Final deliverable}\label{Impl}

In this chapter you cover the final or ``ultimate'' version of your project. It will show the final BNF, the final GUI, the architecture (which should be MVVM or MVC) that includes UML diagrams, additional algorithms if not already included in the previous sprint sections.

\section{Final BNF}

\section{Final GUI}

\section{Code architecture}

\section{Algorithms}

Algorithms can be described in this chapter if not already covered in previous sections. Pseudo-code is preferred over code snippets. If you use the latter then make sure it is well commented inside the code or via the figure caption. 

\begin{algorithm}[th]
\caption{ The Newton-Raphson method }
\begin{algorithmic}[1]
\STATE Initialise root based on estimate
\STATE Set stop criterion
\\ \texttt{const double error = 0.000001;}
\WHILE {stop criterion not met}
	\STATE Compute f(root)
	\STATE Compute f'(root)
	\STATE root := root - f(root)/f'(root)
\ENDWHILE
\end{algorithmic}
\end{algorithm}


Note that code snippets or lists of crucial programming code or large UML diagrams should go in Appendix \ref{app:other} (or further appendices).

\subsection{Testing}

Describe what testing you have done on the interpreter (lexer, parser and execution), GUI and GUI-Interpreter communication, plotting, etc. Table \ref{Table2} in Appendix \ref{app:test} should be completed to do basic arithmetic expression tests.


\chapter{Discussion, conclusion and future work}

Briefly discuss  your achievements and put them in perspective with the MoSCoW analysis you specified in Table \ref{Table1}. Also discuss future developments and how you see the deliverable improving if more time could be spent. Note that this section should not be used as a medium to vent frustrations on whatever did not work out (group issues, not enough time, illness, etc.) as this should be dealt with separately - keep it professional!


\bibliographystyle{apalike}
\raggedright
\bibliography{References}


\appendix
\chapter{Contributions}

State here the \% contribution to the project of each individual member of the group and describe in brief what each member has done (if this corresponds to particular sections in the report then please specify these).

\chapter{Testing}
\label{app:test}
\section{Arithmetic expression testing}

\begin{table}[h]
\caption{Arithmetic expression tests. Note that floating pointing values are accurate to three decimal places for the fractional part. ResE is expected result and ResA is actual result. \\}
\begin{tabular}{|p{1.8in}|p{0.5in}|p{0.4in}|p{0.6in}|p{1.4in}|} \hline
Expression & ResE & ResA& Pass/Fail & Action/comment \\ \hline \hline
$5*3+(2*3-2)/2+6$ & 23 &  &  &  ... \\ \hline
$9-3-2$ & 4 & & & left assoc.\  \\ \hline
$10/3$ & 3 & & & int division  \\ \hline
$10/3.0$ & 3.333 & & & float division \\ \hline
$10\%3$ & 1 & & & \\ \hline
$10 - -2$ & 12 & & & unary minus\\ \hline
$-2 + 10$ & 8 & & & \\ \hline
$3*5\verb|^|(-1+3)-2\verb|^|2*-3$ & 87 & & & power test \\ \hline
$-3\verb|^|2$ & -9(*) or 9 & & & precedence \\ \hline
$-7\%3$ & 2(*) or -1 & & & precedence (*)Python\\ \hline
$3*5\verb|^|(-1+3)-2\verb|^|-2*-3$ & 75.750 or 75 & & & \\ \hline
$3*5\verb|^|(-1+3)-2.0\verb|^|-2*-3$ & 75.750 & & & \\ \hline
$(((3*2--2)))$ & 8 & & & \\ \hline 
$(((3*2--2))$ & Error & & & syntax error \\ \hline
$-((3*5-2*3))$ & -9 & & &  minus expression \\ \hline
$x = 3; (2*x)-x\verb|^|2*5$ & -39 & & & var assign \\ \hline
$x = 3; (2*x)-x\verb|^|2*5/2$ & -16 & & & \\ \hline
$x = 3; (2*x)-x\verb|^|2*(5/2)$ & -12 & & & \\ \hline
$x = 3; (2*x)-x\verb|^|2*5/2.0$ & -16.5 & & & \\ \hline
$x = 3; (2*x)-x\verb|^|2*5\%2$ & 5 & & &  \\ \hline
$x = 3; (2*x)-x\verb|^|2*(5\%2)$ & -3 & & &  \\ \hline
... & ... & ... & ... & ... \\ \hline
\end{tabular}
\label{Table2}
\end{table}

\section{GUI testing}

\section{Plot testing}

\chapter{Other stuff}
\section{MoSCoW table}
\label{appendix:moscow}
\begin{table}[h]
\caption{MoSCoW}
\begin{center}
\begin{tabular}{|p{1in}|p{2in}|p{2.5in}|} \hline
Priority & Task & Comments \\ \hline \hline
\multirow{3}{1in}{Must}
& Solve arithmetic expressions & Works with ints and floats, processing is left-associative, BODMAS/BIDMAS rules apply \\ \cline{2-3}
& Have basic GUI & Input textbox, output textbox and help functionality \\ \cline{2-3}
& Have variable assignment & Assignment token, Grammar change, Stored in a symbol table \\ \cline{2-3}
& Be tested & Unit and functionally tested \\ \cline{2-3}
& Have basic plotting & Plotting area in GUI window to visualise mathematical functions: lines and polynomials \\ \hline \hline
\multirow{3}{1in}{Should}
& AST Parser & Parser constructs an Abstract Syntax Tree \\ \cline{2-3}
& Functions & Works with built-in functions such as cos, sin, tan, exp, log  \\ \cline{2-3}
& Multiple plots & User can plot multiple equations and clear the plotting area \\ \hline \hline
\multirow{3}{1in}{Could}
& For loops & \\ \cline{2-3}
& Conditionals & If statements \\ \cline{2-3}
& Rational numbers & \\ \cline{2-3}
& Complex numbers & \\ \cline{2-3}
& Rational equations plotting & Plot equations like 1/x that produce more than 1 line \\ \hline \hline
\multirow{3}{1in}{Will not}
& Compiler & Too much work to implement a compiler in a given timeframe \\ \cline{2-3}
& Static typing & Making our language statically typed requires a very strong typing system in place which is too hard to implement in a give timeframe \\ \hline
\end{tabular}
\label{Table1}
\end{center}
\end{table}
\end{document}

